\section{UNO}
   En este caso se cambiara desde mi tablet para generar mayor utilizacion de esta y cuidar mi espalda.
			aca se hara el uso de codigo latex, como por ejemplo:

\lipsum(5-10)

\subsection{Caracteristicas}
\begin{itemize}
\item uno
\item dos
\item tres
\end{itemize}
fin\\
\\
aca se podra continuar una nueva linea\\
\\
\newpage
Nueva pagina
\begin{figure}
   \centering
   \includegraphics[scale=0.5]{H:/Toshi/img/ana1}
\end{figure}

Luego de una analisis sincero y directo se ha detectado que la inversion de cuyes no es algo que verdaderamente me agrade realizar, porque busco desarrollar mis potenciales para tener un mejor ingreso haciendo algo, aunque suene cursi, que me apasione, por este motivo el siguiente paso para la tan ansiada inversión que necesito, con mucha urgencia, realizar es dictar el curso de Latex y posteriormente cursos como formulación de proyectos, analisis de datos con R y Latex, asi como no descuidar la formulación de proyectos de inversión publicos y privados. Para esto mi camino debe ser el perfeccionamiento en Latex, formulación de proyectos, estadistica con softwares y base de datos (SQL). Este es tu camino, por favor tomatelo en serio, ya estas viejo para estar divagando y perdiendo el tiempo, te queda solo 8 años para que hayas conseguido lo que necesitas para asegurar tu vejez y tres para ser padre. Ahora es el momento de avanzar por un objetivo claro; se responsable de tus acciones y hazte cargo de estas, todas deben ir orientadas a lo que de ahora en adelante te planteaste, todas tus fortalezas se centran en lo que tienes como objetivo,  porque tienes el talento y las cualidades, confio en ti, se que lo haras y seras muy bueno en esto, ya empezaste el camino, no lo dejes, te adoro con todo mi corazón amor mio e infinito. Son personas como tu las que este pais necesita para que este país crezca bien y de mejor forma. Vuela alto mas alto que los águilas, no temas a la grandeza de tus alas.

se ingresara las preguntas de examenes obtenidos




1.en el marco del sistema nacional de programación Multianual y gestión de inversiones, un programa de inversion puede contener.
\begin{itemize}
			\item proyectos de inversion, estudios, IOARR y componentes de administracion o gestión 
			\item Proyectos piloto, IOARR, estudios, proyectos de inversión y gastos de operacion y mantenimiento
			\item Proyectos de inversion, IOARR, proyectos piloto o alguna otra intervención relacionada directamente a la consecucion del objetivo del programa de inversión 
			\iterm IOARR, proyectos piloto, componentes de administracion o gestiion, proyectos de inversion, estudios o alguna otra intervencion relacionada directamente a la consecucion del objetivo del programa de inversion
			\item ninguna de las anteriores 

\end{itemize}


3. respecto de los lineamientos para la identificación y registro de las IOARR, señale la alternativa que considere incorrecta

\begin{itemize}

			\item una IOARR es una intervencion puntual sobre uno o mas activos estrategicos que integran una unidad productora en funcionamiento
			\item La definicion de IOARR nos permite distinguirlas en dos grupos de inversiones: IOARR con enfoque de unidad productora e IOARR con enfoque de activo estrategico 
			\item Un activo califica como activo estratégico cuando directa o indirectamente se constituye en un factor limitante de la capacidad de producción del servicio que brinda una unidad productora
			\item Si para optimizar la unidad productora se reemplazan, reparan o incrementan los activos, no es necesario identificar multiples IOARR porque es posible incluirlas dentro de la optimizacion
			\item La inversión masiva se refiere a la aplicacion de las IOARR de optimizacion, de reposicion, dee rehabilitacion y ampliacion amrginal de la edificacion u obra civil en varias unidades productoras en un mismo registro 
		

\end{itemize}

4. En los siguientes enunciados, identifique la alternativa que considere falsa

begin{itemize} 
			\item El nivel de complejidad de un proyecto de inversion se define en funcion a: 1) el nivel de riesgo o incertidumbre de los resultados del proyecto de inversion, y 2) el valor o magnitud del monto estimado de la inversion del proyecto
			\item Cuando la operacion y mantenimiento de los proyectos de inversion se encuentra a cargo de una entidad distinta a la que pertenece la unidad formuladora que formula el proyecto de inversion, debe contarse con la opinión de aquella sobre la prevision de los gastos de operacion y mantenimiento
			\item El orden de prelacion de las inversiones priorizadas en la cartera de inversiones segun su avance  en el ciclo de inversión se determina en el módulo  del programa multianual de inversiones 
			\item El inventario de activos reune informacion relevante y actualizada sobre los activos generados por la ejecucion de las inversiones, principalmente respecto de su stock, estado situacional y capacidad de produccion de servicios 
			\item Todas las anteriores son verdaderas

end{itemize}

5. Respecto de las inversiones, del ambito de responsabilidad funcional, a formularse o aprobarse y a ejecutarse, la verificacion de que esten alineadas con los objetivos priorizados y las metas respecto al cierree de brechas de infraestructura o de acceso a servicios establecidos en la programación  Multianual  de inversiones y cumplan con los criterios de priorizacion aprobados por el sector es funcion de:

begin{itemize}
				\item La unidad ejecutora de inversiones
				\item La unidad formuladora
				\item El organo resolutivo
				\item La oficina de programacion multianual de inversiones
				\item La direccion general de programacion multianual de inversiones


end{itemize}

6. En el marco del SNPMyGI, los factores de produccion de una unidad productora de bienes o servicios se clasifican en:


begin{itemize}
				\item infraestructura natural; intangibles; terreno; activos estrategicos e infraestructura
				\item Terreno; infraestructura; equipo, mobiliario y vehiculo; intangibles; y recursos financieros
				\item infraestructura; terreno;v equipo, mobiliario y vehiculo; intangibles; e infraestructura natural
				\item infraestructura; equipo, mobiliario y vehiculo; intangibles; e infraestructura natural
				\item ninguna de las anteriores


end{itemize}


7. En los siguientes enunciados, identifique la alternativa que considere correcta:

begin{itemize}
				\item La fase de ejecucion del ciclo de inversión comprende la elaboracion del expediente tecnico o documenti equivalente y la ejecucion fisica y financiera respectiva
				\item Las IOARR no aplican sobre activos vinculados al factor de produccion infraestructura natural
				\item La unidad formuladora registra el proyecto de inversion en el banco de invertsionees, asi como el resultaedoo de la evaluaciion realizada
				\item La infraestructura natural es la red de espacios naturales que conservan los valores y funciones de los ecosistemas, proveyendo servicios ecosistemicos
				\item Todas las alternativas anteriores son correctas


end{itemize}

8. Respecto a la consistencia del prrograma maultianual de inversiiones con elproyecto/ley de presupuesto del sector publico, identifique la alternativa correcta:

begin{itemize}
				\item La direccion General de Presupuesto Publico remite a la Direccion General de Programacion Multianul de Inversiones la informacion correspondiente a la programación y formulacion presupuestaria de las inversiones establecidas en el proyecto de ley y en la ley anual de presupuesto efectuada por los sectores, gobiernos regionales y gobiernos locales a travees de sus pliegos para evaluar la consistencia con el programa multianual de inversiones 
				\item La Direccion General de Programacion Multianual de Inversiones evalúa la consistencia del programa multianual de inversiones evalúa la consistencia del programa multianual de inversiones y lo actualiza en coordinación con los sectores, gobierno regional y gobierno local, considerando las asignaciones presupuestales a la inversiones en el proyecto de ley y en la ley anual de presupuesto 

				\item En el caso de las empresas públicas bajo el ambito del FONAFE, incluido ESSALUD, la consistencia se realiza con el presupuesto de inversiones consolidado remitido a la Direccion General de Programacion Multianual de Inversiones por el director ejecutivo del FONAFE
				\item Las actualizaciones realizadas a la cartera de inversiones del programa multianual de inversiones por efecto de la consistencia, son aprobadas mediante un informe técnico emitido por la oficina de programación multianual de inversiones de los sectores, gobierno regional y gobierno local, el cual se adjunta en el aplicativo informatico Modulo del Programa Multianual de Inversiones
				\item Tofas las alternativas anteriores son correctas 


end{itemize}

9. Es un proyecto especial de la PCM oficializada a traves del D.S. 084-2023-pcm denominado

begin{itemize}
			\item Legado
 			\item IPD
			\item Juegos panamericanos juegos parapanemericanos
end{itemize}

10. En el marco del SNPMyGI, los servicios que pueden ser intervenidos a traves de inversiones se encuentran identificados dentro de:

begin{itemize}
			\item La guia general para la identificación, formulación y evaluación de proyectos de inversión 
 			\item Los lineamientos para la identificación y registro de las IOARR
			\item El Anexo 02: clasificador de responsabilidad funcional del sistema del SNPMyGI, de la directiva general de dicho sistema administrativo 
			\item Los lineamientos para el inventario de unidades productoras y activos estrategicos 
			\item ninguna de las anteriores 
end{itemize}

11. En la evaluación social, para la estimación de los indicadoras de resntabilidad social de un proyecto de inversion se aplican cualquiera de las metodologias siguientes: 

begin{itemize}
			\item Costo - beneficio; costo - eficacia; y tasa de interes
 			\item Costo - efectividad; costo - eficacia; y tasa social de descuento
			\item Costo - beneficio; costo - efectividad o costo - eficacia
			\item Costo - eficacia; costo - beneficio; y nivel de ingresos del proyecto de inversion
			\item Ninguna de las anteriores 
end{itemize}

12.  El poder ejecutivo tiene la rectoria de los sistemas administrativos, con excepcion del:

begin{itemize}
			\item Sistema nacional de control 
 			\item sistema nacional de defensa judicial del estado
			\item sistema nacional de modernización de la gestión pública 
			\item sistema nacional de abastecimiento 
			\item sistema nacional de gestión de recursos humanos 
end{itemize}

13. De acuerdo a la Ley N 287|6 Ley de control interno de las entidades del estado, el sistema de control interno es el:

begin{itemize}
			\item conjunto de acciones, actividades, planes, politicas, normas, registros, organización, procedimiento y métodos 
 			\item conjunto de acciones, planes, politicas, normas, registros, procedimiento y metodos
			\item conjunto de actividades, planes, politicas, normas, registros, procedimiento y métodos 
			\item ninguna de las anteriores 
end{itemize}

14. en referencia a los criterios de priorizacion sectoriales, identifique la alternativa que considere incorrecta 

begin{itemize}
			\item El organo resolutivo del sector aprueba las brechas identificadas y los criterios de priorizacion de las inversiones a ser aplicadas en la elaboracion de su programa multianual de inversiones, de acuerdo a las medidas sectoriales definidas por los sectores 
 			\item La OPMI del sector propone al órgano resolutivo los criterios de priorizacion de la cartera de inversiones y brechas identificadas a considerarse en el programa multianual de inversiones sectorial, los cuales deben tener en consideracion los planes nacionales sectoriales establecidos en el planeamiento estrategico de acuerdo al SINAPLAN y ser concordantes con las proyecciones del marco macroeconomico multianual 
			\item Los criterios de priorizacion son parte del programa multianual de inversiones junto con el diagnóstico de la situación de las brechas de infraestructura y/o de acceso a servicios, y la cartera de inversiones bajo la responsabilidad funcional de un sector, o a cargo de un gobierno regional, gobierno local o empresa pública 
			\item Los criterios de priorizacion sectoriales obligatorios son: criterio de priorizacion de cierres de brechas, criterio de priorizacion de alinamiento al planeamiento estrategico y el criterio de priorizacion de ejecutabilidad  
			\item Los criterios de priorizacion son parametros tecnicos definidos por las oficinas de programacion multianual de inversines, que tienen el proposito de establecer una jerarquia u orden de importancia de una inversion en la cartera de inversiones del programa multianual de inversiones de una entidad
end{itemize}

15. ¿cuales de los siguientees requisitos se deben cumplir para otorgarse la declaracion de viabilidad de un proyecto de inversión?

begin{itemize}
			\item La entidad ha cumplido con los procesos y procedimeintos del SNPMyGI; el objetivo central del proyecto de inversion se encuentra alineado al cierre de brechas de infraestructura o de acceso a servicios; el proyecto de inversion no esta sobredimensionado respecto de la demanda prevista y sus deneficios sociales no estan sobredimensionados
 			\item La intervencion guarda correspondencia con la definicion de proyecto de inversion; se cautela la sostenibilidad del proyecto de inversion, que incluye asegurar su operacion y mantenimiento; la ficha tecnica o el estudio de preinversion del proyecto de inversion ha sido formulado considerando metodologias de formulacion y evaluación ex ante de proyectos de inversión aprobadas por la DGPMI y por el sector, segun  
			\item No se trata de un proyecto de inversion fraccionado ni duplicado; la unidad formuladora tiene competencias legales para formular y declarar la viabilidad del proyecto de inversion; la ficha tecnica del proyecto de inversion cuenta con la opinion favorable de la OPMI correspondiente
			\item No se trata de un proyecto de inversión fraccionado ni duplicado; la ficha tecnica o el estudio de preinversion del proyecto de inversion ha sido elaborado considerando los parametros y normas técnicas sectoriales y los parametros de evaluacion social; la unidad formuladora tiene competencias legales para formular y declarar la viabilidad del proyecto de inversion 
			\item Las alternativas a), b) y d) son correctas
end{itemize}

16. Diga que funcion y/o funciones pertenece al presidnete del consejo de ministros, segun la ley organica del poder ejecutivo
begin{itemize}
			\item Presidir y dirigir la comision interministerial de asuntos economicos y financieros - CIAEF, la comision interministerial de asuntos sociales - CIAS; y las demas comisiones interministeriales cuando corresponda
 			\item Formular, aprobar y ejecutar las politicas nacionales de modernizacion de la administracion publica y las relacionadas con la estructura y organización del estado; asi como coordinar y dirigir la modernizacion del estado
			\item Coordinar la planificación estrategica concertada en el marco del sistema nacional de planeamiento estrategico 
			\item Informar anualmente al congreso de la republica sobre los avances en el cumplimiento del plan nacional de accion por la infancia, de la ley de igualdad de oportunidades, el plan nacional de derechos humanos y otros de acuerdo a ley
			\item Todas las anteriores son funciones de la PCM
end{itemize}

17. ¿Cual de los siguientes son organismos reguladores de la PCM?

begin{itemize}
			\item Organismo supervisor de la inversion en infraestructura de transporte de uso publico, organismo supervisor de la inversion privada en in telecomunicaciones, superintendencia nacional de servicios de saneamiento, organismo supervisor de la inversion en infraestructura de transporte de uso público 
 			\item organismo supervisro de la inversion en infraestructura de transporte de uso publico, centro nacional de planeamiento estrategico, superintendencia nacional de servicios de saneamiento, organismo supervisor de inversion en infraestructura de transporte de uso público 
			\item organismo supervisor de la inversion en infraestructura de transporte de uso público,  organismo supervisor de la inversión privada en telecomunicaciones,  superintendencia nacional de servicios de saneamiento,  comision nacional para el desarrollo y vida sin drogas
			\item organismo supervisor de la inversion en infraestructura de transporte de uso publico, organismo supervisor de la inversion privada en telecomunicaciones, consejo nacional de ciencia, tecnologia e innovacion tecnologica, autoridad nacional de infraestructura 
			\item ninguna de las anteriores
end{itemize}

18. respecto de las funciones del organo resolutivo del sector, identifique la alternativa incorrecta
begin{itemize}
			\item Aprobar el programa multianual de inversiones del sector, asi como las modificaciones de los objetivos priorizados e indicadores establecidos en el programa multianual de inversiones 
 			\item Designar a la oficina de programación multianual de inversiones del sector
			\item Designar al responsable de la oficina de programacion multianual de inversiones de acruerdo con el perfil profesional que establece la direccion general de programacion multianual de inversiones
			\item Autorizar la elaboracion de expedientes tecnicos o documentos equivalentes de proyectos de inversion, asi como su ejecucion, cuando estos hayan sido declarados viables mediante fichas tecnicas. Dicha funcion puede ser obejto de delegación 
			\item presidir el comite de seguimeinto de inversiones del sector, no siendo esta funcion objeto de delegación 
end{itemize}

19. En los siguientes enunciados, identifique la alternativa que considere falsa

begin{itemize}
			\item La metodologia de evaluacion social de costo efectividad o costo eficacia mide la relacion entre los recursos empleados y los resultados o impactos alcanzados
 			\item para declarar la viabilidad de un programa de inversion se debe haber declarado viable aquellos proyectos de inversion que representen por lo menos el treinta por ciento (30%) del monto de inversion total a precios de mercado
			\item La consistencia es la accion por la cual la unidad formuladora corrobora que la concepcion tecnica permanece inalterada y que se cumplen con las condiciones de dimensionamiento y viabilidad del proyecto de inversión 
			\item La viabilidad de un proyecto de inversion, requisito previo a la fase de ejecucion, se aplica a un proyecto de inversion cuando a traves de la ficha tecnica o estudio de preinversion ha evidenciado estar alineado al cierre de brechas de infraestructura o de acceso a servicios, tener una contribución al bienestar de la poblacion beneficiaria y al resto de la sociedad en general y que dicho bienestar sea sostenible durante el funcionamiento del proyecto de inversión 
			\item Todas las alternativas anteriores son falsas
end{itemize}

20. en el marco del SNPMyGI, los tipos de inversion no previstas para modificaciones de la cartera de inversiones del programa multianual de inversiones son:

begin{itemize}
			\item Ampliacion y migración 
 			\item Adelantada y continuidad
			\item Nueva, ampliación, adelantada, migración y continuidad
			\item ampliación, mejoramiento, nueva, adelantada y migracion
			\item ninguna de las anteriores 
end{itemize}


\textbf{examen del mininter}

1. Es el conjunto de recursos o factores productivos (infraestructura, equipos, personal, organización,  capacidades de gestion, entre otros) que, articulados entre si, tienen la capacidad de proveer bienes o servicios a la poblacion objetivo. Constituye el producto generado o modificado por un poryecto de inversión 

begin{itemize}
			\item expediente tecnico
 			\item servicios
			\item unidad productora
			\item los recursos destinados a la inversion deben procurar el mayor impacto en la sociedad 
			\item ninguna de las anteriores 
end{itemize} 

2. segun lo descrito para la fase de formulacion y evaluacion, la formulacion se realiza:

begin{itemize}
			\item Solo a traves de una ficha tecnica
 			\item Solo a traves de un estudio de preinversion 
			\item a traves de una ficha tecnica y solo en caso de proyectos que tengan alta complejidad se requiere un nivel de estudio de preinversion (perfil)
			\item conforme lo que norme el sector
			\item ninguna de las anteriores
end{itemize}

3. solo pueden recibir transferencias del GN, los GR o GL cuyas inversiones, seleccione una:

begin{itemize}
			\item sean gestionadas personalmente por los mismos gobernadores o alcaldes
 			\item cumplan con los criterios de priorizacion que aprueben los sectores
			\item cumplan con el unico criterio de cierre de brechas 
			\item cuenten con expediente tecnico aprobado
			\item ninguna de las anteriores
end{itemize}

4. Completar -------------------- que se elaboren y evaluen en el marco del SNPMyGI tienen caracter de declaracion jurada y su veracidad constituye estricta responsabilidad de la UF, siendo aplicables las responsabilidades que determine la contraloria general de la republica y la normativa vigente 
begin{itemize}
			\item las fichas
 			\item las fichas tecnicas
			\item las fichas estandar
			\item las fichas tecnicas y los estudios de preinversion a nivel de perfil
			\item los estudios de preinversion 
end{itemize}

5. Atendiendo a la forma de un indicador de desempeño, de las siguientes posibilidades ¿Cual podria ser un indicador de brecha de servicio en el marco del Invierte.pe?
begin{itemize}
			\item numero de locales educativos con educacion primaria
 			\item numero de locales educativos con educacion primaria que contiene capacidad instalada adecuada
			\item numero de locales educativos con educacion primaria que contiene capacidad instalada inadecuada
			\item  porcentaje de locales educativos con educación primaria que contiene capacidad instalada inadecuada
			\item porcentaje de locales educativos con educacion primaria que contiene capacidad instalada adecuada
end{itemize}

6. ¿Quien aprueba los indicadores de brechas y los criterios de priorizacion a ser aplicados en la fase de programacion multianual de inversiones para los tres niveles de gobierno?

begin{itemize}
			\item cada uno de los organos resolutivos (nacional, regional, local)
 			\item solo el organo resolutivo del sector del GN
			\item cada una de las UF de las entidades
			\item el MEF a traves de la DGPMI
			\item ninguna de las  anteriores 
end{itemize}

7. caso: de retorno de un operativo policial, el unico patrullero de una comisaria sufre una volcadura ocasionada por la maniobra temeraria del conductor de un trailer que iba por el mismo camino. El vehiculo queda con daños estructurales y funcionales; ademas, los equipos de comunicaciones y los de tecnologia (laptop para el servicio inteligente) sufren daños severos. Esta era la unica unidad de patrullaje con la que contaba para el servicio de seguridad ciudadana por lo que debe contarse con una patrulla de iguales caracteristicas a la brevedad. ¿que tipo de IOARR recomendaria formular para resolver el  caso?

begin{itemize}
			\item no recomendaria ningun IOARR pues se trata de un proyecto de inversión 
 			\item ampliacion marginal de adquisición anticipada de terrenos
			\item ampliación marignal de servicio 
			\item reposicion
			\item rehabilitacion
end{itemize}

8. Si el monto de inversion de un proyecto de inversion se esta calculando alrededor de las 100 UIT ¿que documento tecnico corresponde elaborar? 

begin{itemize}
			\item Perfil simplificado
 			\item perfil refrozado
			\item ficha tecnica simplificada
			\item ficha tecnica estandar
			\item ninguna de las anteriores 
end{itemize}

9. ¿que procesos se encuentran en el ambito de la discrecionalidad de los funcionarios por tratarse de procesos necesarios para la toma de decisiones referentes a la inversión?

begin{itemize}
			\item la programacion multianual
 			\item la formulación 
			\item la ejecucion de inversiones 
			\item la implementacion de modificaciones durante la ejecucion
			\item todas las anteriores
end{itemize}

10. La ejecucion fisica de las inversiones se inicio luego de la aprobacion del expediente tecnico o documento equivalente segun corresponda, siendo responsabilidad de la UEI efectuar los registros que correspondan en el Banco de inversiones 

begin{itemize}
			\item Elaboracion del perfil
 			\item elaboracion de las fichas tecnicas
			\item elaboracion del expediente tecnico o documento equivalente 
			\item culminacion del expediente tecnico o documento equivalente 
			\item aprobacion del expediente tecnico o documento equivalente 
end{itemize}
 

\textbf{primer examen de la PCM}



1. La fase de funcionamiento del ciclo de inversion, comprende:

begin{itemize}
			\item La operacion y mantenimiento de los activos generados con la ejecución de la inversion y la provision de los servicios implementados con dicha inversion 
 			\item la formulacion del perfil
			\item la elaboracion del expediente tecnico 
			\item todas las anteriores
end{itemize}

2,en la fase de ejecucion del ciclo de inversion, la consistencia esta relacionada

begin{itemize}
			\item solamente a la viabilidad del proyecto 
 			\item solamente a la aprobación del expediente tecnico 
			\item a la concepcion tecnica, el dimensionamiento y viabilidad del proyecto de inversion 
			\item al sistema de seguimiento de infraestructura 
			\item ninguna de las anteriores
end{itemize}

3. respecto de las modificaciones de la cartera de inversiones del programa multianual de inversiones, señale cual es la alternativa incorrecta

begin{itemize}
			\item
 			\item
			\item
			\item
			\item
end{itemize}



begin{itemize}
			\item
 			\item
			\item
			\item
			\item
end{itemize}



begin{itemize}
			\item
 			\item
			\item
			\item
			\item
end{itemize}



begin{itemize}
			\item
 			\item
			\item
			\item
			\item
end{itemize}



begin{itemize}
			\item
 			\item
			\item
			\item
			\item
end{itemize}



begin{itemize}
			\item
 			\item
			\item
			\item
			\item
end{itemize}



begin{itemize}
			\item
 			\item
			\item
			\item
			\item
end{itemize}



begin{itemize}
			\item
 			\item
			\item
			\item
			\item
end{itemize}



begin{itemize}
			\item
 			\item
			\item
			\item
			\item
end{itemize}




begin{itemize}
			\item
 			\item
			\item
			\item
			\item
end{itemize}




begin{itemize}
			\item
 			\item
			\item
			\item
			\item
end{itemize}




begin{itemize}
			\item
 			\item
			\item
			\item
			\item
end{itemize}



begin{itemize}
			\item
 			\item
			\item
			\item
			\item
end{itemize}



begin{itemize}
			\item
 			\item
			\item
			\item
			\item
end{itemize}



















