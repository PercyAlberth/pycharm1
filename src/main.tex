%! Author = PERCY
%! Date = 24/12/2023

% Preamble
\documentclass[11pt]{article}
\usepackage{ipsum}
\usepackage{lipsum}
\usepackage[utf8]{inputenc}
\usepackage{float}
\usepackage{graphicx}


\begin{document}
    Changed content ipsum dolor sit amet, consectetur adipiscing elit, sed do eiusmod tempor incididunt ut labore et dolore
    magna aliqua. Enim facilisis gravida neque convallis. Id cursus metus aliquam eleifend mi in nulla. Suspendisse
    ultrices gravida dictum fusce ut placerat orci nulla. Placerat in egestas erat imperdiet. Et sollicitudin ac orci
    phasellus. Ut porttitor leo a diam sollicitudin tempor. Eget nulla facilisi etiam dignissim diam quis enim.
    Ornare massa eget egestas purus viverra accumsan. Cursus in hac habitasse platea dictumst quisque sagittis purus
    sit. Nec ultrices dui sapien eget mi proin sed libero enim. Eget est lorem ipsum dolor sit amet consectetur
    adipiscing elit. Nulla facilisi etiam dignissim diam quis enim lobortis scelerisque.

    aca se agregara un nuevo parrafo, para probar git

     \section{UNO}
    Propietario. Te doy la bienvenida ut venenatis tellus in metus. A diam maecenas sed enim. In massa tempor nec feugiat nisl pretium. Lorem
    dolor sed viverra ipsum nunc aliquet bibendum enim facilisis. Dignissim diam quis enim lobortis. Tortor vitae
    purus faucibus ornare suspendisse sed nisi lacus sed. In nisl nisi scelerisque eu ultrices vitae auctor eu augue
    . Amet nulla facilisi morbi tempus. Eu consequat ac felis donec et odio. Nibh cras pulvinar mattis nunc sed
    blandit libero volutpat. Fermentum leo vel orci porta non. Sapien faucibus et molestie ac. Tincidunt dui ut
    ornare lectus. Erat pellentesque adipiscing commodo elit. Risus nullam eget felis eget. Viverra aliquet eget sit
    amet tellus cras adipiscing enim eu. Eget aliquet nibh praesent tristique magna sit amet purus. Ullamcorper
    dignissim cras tincidunt lobortis feugiat vivamus at augue eget. Nunc vel risus commodo viverra maecenas. Lectus
    mauris ultrices eros in cursus.


    Magna etiam tempor orci eu. A condimentum vitae sapien pellentesque habitant morbi tristique senectus et. Fusce
    id velit ut tortor. Sit amet mattis vulputate enim nulla aliquet porttitor lacus. Turpis nunc eget lorem dolor
    sed. Ante metus dictum at tempor commodo ullamcorper a lacus vestibulum. Dolor sit amet consectetur adipiscing
    elit ut aliquam purus. Faucibus et molestie ac feugiat sed lectus vestibulum. Lobortis elementum nibh tellus
    molestie nunc non blandit. Varius morbi enim nunc faucibus a pellentesque sit amet. Eget aliquet nibh praesent
    tristique magna sit amet. Ut sem nulla pharetra diam sit. Scelerisque fermentum dui faucibus in ornare quam. Nunc
    id cursus metus aliquam eleifend mi in. Felis eget velit aliquet sagittis id consectetur. Viverra orci sagittis
    eu volutpat. Amet commodo nulla facilisi nullam vehicula ipsum a arcu. Tellus id interdum velit laoreet id donec
    ultrices tincidunt arcu.

    Nuevo Magna etiam tempor orci eu. A condimentum vitae sapien pellentesque habitant morbi tristique senectus et. Fusce
    id velit ut tortor. Sit amet mattis vulputate enim nulla aliquet porttitor lacus. Turpis nunc eget lorem dolor
    sed. Ante metus dictum at tempor commodo ullamcorper a lacus vestibulum. Dolor sit amet consectetur adipiscing
    elit ut aliquam purus. Faucibus et molestie ac feugiat sed lectus vestibulum. Lobortis elementum nibh tellus
    molestie nunc non blandit. Varius morbi enim nunc faucibus a pellentesque sit amet. Eget aliquet nibh praesent
    tristique magna sit amet. Ut sem nulla pharetra diam sit. Scelerisque fermentum dui faucibus in ornare quam. Nunc
    id cursus metus aliquam eleifend mi in. Felis eget velit aliquet sagittis id consectetur. Viverra orci sagittis
    eu volutpat. Amet commodo nulla facilisi nullam vehicula ipsum a arcu. Tellus id interdum velit laoreet id donec
    ultrices tincidunt final. aca se esta cambiando

\section{Aspectos necesarios para el estudio de mercado}
Para este emprendimiento se tendra que considerar todo lo necesario para que esto pueda funcionar, desde un analisis de la demanda hasta un analisis de la oferta, primero se hara un analisis FODA, motivo por el cual se desarrollara la siguiente:
\subsection{Analisis interno}

\textbf{Fortalezas}
\begin{itemize}
	\item Juventud
	\item Nocion de administracion y organizacion
	\item Ganas de trabajar
	\item Facilidad de obtener financiamiento
	\item Facilidad para el uso de herramientas informaticas para el control y operacion
	\item Facilidad para el uso de redes sociales y posibles aliados estrategicos
    \item Contactos que nos pueden orientar
\end{itemize}

\textbf{Oportunidades}
\begin{itemize}
	\item Demanda en el mercado local debido a fiestas costumbristas
	\item Facilidad en produccion de insumos
	\item Mayor promoción en el consumo de cuy
	\item Demanda por parte del mercado turístico 
	\item Demanda por productos de calidad
\end{itemize}

\textbf{Debilidades}
\begin{itemize}
	\item Dependencia de un inmueble propio para los galpones
	\item Dificultad para armar el presupuesto debido a muchos distractores
	\item Poca disponibilidad de efectivo
	\item Poca experiencia en la crianza de cuyes
	\item Larga distancia para la obtención de insumos para el centro de producción 
\end{itemize}

\textbf{Amenazas}
\begin{itemize}
	\item Desarrollo de enfermedades en los cuyes  
	\item Poca demanda de los consumidores
	\item Precios bajos
	\item Existencia de competencia
    \item Estafa por parte de proveedores e insumos y cuyes
	\item Vulnerabilidad a cambios externos
\end{itemize}

\subsubsection{Estrategias de crecimiento}
Esta es la combinación de las \textbf{fortalezas} y \textbf{oportunidades}, de estas se considerara las siguientes:
\begin{itemize}
    \item Utilizacion de redes sociales para la promocion del producto en fiestas costumbristas y alianza con aliados
    \item Ganas de trabajar para la produccion de alfalfa
    \item Obtencion de financiamiento para atender la demanda local de cuyes
    \item Buena organizacion y control para la produccion de calidad de cuyes
    \item Facilidad de uso de herramientas informaticas para una adecuada atencion a la demanda
\end{itemize}

\subsubsection{Estrategias de defensa}
Esta es la combinación de las \textbf{amenazas} y \textbf{fortalezas}
\begin{itemize}
    \item Uso de herramientas informaticas para el control y organizacion para prevenir enfermedades de loc cuyes
    \item Uso de redes sociales para mayor promocion y prevencion de poca demanda de los consumidores
    \item Uso de redes sociales para una mejor promocion de la calidad y acceder a mercados que paguen buenos precios
    \item Ganas de trabajar, aprender, averiguar y consultar como prevenir estafas futuras
\end{itemize}

\subsubsection{Estrategias de adaptación}
Esta es la combinación de las \textbf{debilidades} y \textbf{oportunidades}
\begin{itemize}
    \item Facilidad de pago el inmueble alquilado gracias a la mayor demanda (ventas) de cuyes
    \item Mayor capacitacion y busqueda de informacion en la crianza de cuyes para obtener un producto de calidad
    \item Generacion de mas efectivo gracias a la demanda de fiestas costumbristas, restaurantes turisticos y mercado local
    \item La facilidad de generacion de insumos es mas barato a comparacion de comprar frecuentemente de terceros
\end{itemize}
\subsubsection{Estrategias de supervivencia}
Esta es la combinación de las \textbf{amenazas} y \textbf{debilidades}
    \begin{itemize}
        \item Mantener precios estables frente a la poca experiencia en la produccion de cuyes
        \item Ofrecer la cantidad disponible en funcion a la produccion respetando la capacidad financiera que tenemos
        \item Poca experiencia para el control de enfermedades, se tendra mayor cuidado y atencion, para esto se implementara un plan de trabajo preventivo
        \item Vulnerabilidad a que ya no nos presten el inmueble frente a esto los galpones deberan ser de facil transporte
    \end{itemize}
\section{Objetivo general}
Mejorar la oferta de cuyes de calidad para el mercado local y extranjero a traves de cuyes de calidad en el departamento de Cusco por los siguientes 5 años.
\section{Objetivos especificos}
\begin{itemize}
    \item Implementar un estudio de mercado para la produccion de cuyes, en una semana
    \item Implementar un presupuesto de funcionamiento e inversion para el inicio de operaciones en un mes
    \item Implementar el proceso de produccion y abastecimiento en la produccion de cuyes en dos meses
    \item Implementar el proceso de promocion y fidelizacion de clientes en una semana
\end{itemize}
\section{Vision}
Somos una empresa sostenible en la produccion de cuyes saludables para el consumo humano; innovadora y creativa frente a nuevos retos del mercado con un personal comprometido con su trabajo y creativo.
    mas
\section{Mision}
Trabajamos para aumentar la oferta de cuyes de calidad, siendo innovadores para el mercado local y extranjero garantizando la salud de los consumidores.
\end{document}
